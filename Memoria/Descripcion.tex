% Descripción de la aplicación incluyendo objetivos

En este trabajo se ha desarrollado un sistema de domótica distribuido orientado a un entorno doméstico sencillo: una puerta de acceso y un punto de iluminación. El sistema toma decisiones automáticamente a partir de sensores de distancia (ultrasonidos) y un sensor de luz (LDR), y además permite forzar manualmente los estados desde aplicaciones externas vía MQTT (por ejemplo, un dashboard web o Node-RED).

El público objetivo es principalmente docente: estudiantes de la asignatura de Internet de las Cosas que necesiten un ejemplo completo de arquitectura IoT (sensores, pasarela, protocolo de campo, broker MQTT y aplicaciones cliente). No obstante, la solución es extensible a escenarios reales de monitorización y control de acceso en pequeña escala.

\section*{Objetivos}

Los objetivos principales del proyecto son:

\begin{itemize}
  \item Diseñar y desplegar una arquitectura IoT completa que conecte sensores físicos, un nodo actuador y una pasarela basada en Raspberry Pi usando LoRa como red de campo y MQTT como capa de integración.
  \item Implementar un protocolo maestro--esclavo ligero para la lectura y configuración de sensores (ultrasonidos y LDR) y un formato de tramas LoRa con direcciones, identificadores de mensaje y ACK de aplicación.
  \item Integrar un broker MQTT (Mosquitto) y un puente serie--MQTT que permita exponer los datos de los sensores y controlar el actuador desde aplicaciones externas de forma desacoplada.
  \item Desarrollar una interfaz web sencilla que visualice en tiempo real el estado de luz y puerta, y que permita enviar comandos manuales reutilizando la infraestructura MQTT existente.
\end{itemize}

Desde el punto de vista funcional, el sistema ofrece:

\begin{itemize}
  \item Encendido y apagado automático de la luz en función del nivel medido por la LDR (umbrales configurados en el maestro).
  \item Apertura y cierre de una puerta simulada mediante un servo, a partir de la detección de presencia por dos sensores ultrasónicos.
  \item Publicación de los estados lógicos (luz y puerta) en topics MQTT para monitorización y análisis.
  \item Posibilidad de forzar manualmente los estados desde la red MQTT sin necesidad de acceder físicamente a los nodos LoRa.
\end{itemize}
