% Descripción de la aplicación incluyendo objetivos

En este trabajo se ha desarrollado un sistema de domótica distribuido orientado a un entorno doméstico sencillo: una puerta de acceso y un punto de iluminación. El sistema toma decisiones automáticamente a partir de sensores de distancia (ultrasonidos) y un sensor de luz (LDR), y además permite forzar manualmente los estados desde aplicaciones externas vía MQTT (por ejemplo, un dashboard web o Node-RED).

El público objetivo es principalmente docente: estudiantes de la asignatura de Internet de las Cosas que necesiten un ejemplo completo de arquitectura IoT (sensores, pasarela, protocolo de campo, broker MQTT y aplicaciones cliente). No obstante, la solución es extensible a escenarios reales de monitorización y control de acceso en pequeña escala.

\section*{Contexto y Motivación}

El Internet de las Cosas (IoT) ha revolucionado la forma en que interactuamos con nuestro entorno, permitiendo la automatización de tareas cotidianas y la monitorización remota de espacios. En el ámbito doméstico, los sistemas de domótica ofrecen beneficios tangibles:

\begin{itemize}
  \item \textbf{Eficiencia energética}: encender luces solo cuando es necesario reduce el consumo eléctrico.
  \item \textbf{Comodidad}: puertas que se abren automáticamente al detectar presencia.
  \item \textbf{Seguridad}: monitorización remota del estado del hogar.
  \item \textbf{Accesibilidad}: control desde dispositivos móviles para personas con movilidad reducida.
\end{itemize}

La tecnología LoRa (Long Range) resulta especialmente adecuada para este tipo de aplicaciones por su largo alcance (hasta varios kilómetros en condiciones ideales), bajo consumo energético y capacidad de penetración en interiores. Combinada con MQTT, un protocolo de mensajería ligero diseñado para dispositivos con recursos limitados, permite construir sistemas escalables y flexibles.

\section*{Objetivos}

Los objetivos principales del proyecto son:

\begin{itemize}
  \item Diseñar y desplegar una arquitectura IoT completa que conecte sensores físicos, un nodo actuador y una pasarela basada en Raspberry Pi usando LoRa como red de campo y MQTT como capa de integración.
  \item Implementar un protocolo maestro--esclavo ligero para la lectura y configuración de sensores (ultrasonidos y LDR) y un formato de tramas LoRa con direcciones, identificadores de mensaje y ACK de aplicación.
  \item Integrar un broker MQTT (Mosquitto) y un puente serie--MQTT que permita exponer los datos de los sensores y controlar el actuador desde aplicaciones externas de forma desacoplada.
  \item Desarrollar una interfaz web sencilla que visualice en tiempo real el estado de luz y puerta, y que permita enviar comandos manuales reutilizando la infraestructura MQTT existente.
\end{itemize}

\section*{Funcionalidades del Sistema}

Desde el punto de vista funcional, el sistema ofrece:

\begin{itemize}
  \item \textbf{Control automático de iluminación}: encendido y apagado de la luz en función del nivel medido por la LDR. Cuando el sensor detecta oscuridad (valor inferior a 500), se enciende el LED; cuando hay luz suficiente, se apaga.
  \item \textbf{Control automático de puerta}: apertura y cierre de una puerta simulada mediante un servomotor SG90. Los sensores ultrasónicos SRF01 y SRF02 detectan la presencia de objetos a menos de 100~cm, lo que activa la apertura.
  \item \textbf{Publicación MQTT}: los estados lógicos (luz=0/1, puerta=0/1) se publican en topics MQTT para monitorización, registro histórico e integración con otras aplicaciones.
  \item \textbf{Control manual}: posibilidad de forzar los estados desde cualquier cliente MQTT (dashboard web, mosquitto\_pub, Node-RED, Home Assistant) sin necesidad de acceder físicamente a los nodos.
\end{itemize}

\section*{Casos de Uso}

A continuación se describen los principales casos de uso del sistema:

\begin{table}[H]
\centering
\caption{Casos de uso del sistema de domótica.}
\label{tab:casos-uso}
\begin{tabular}{lp{8cm}}
\toprule
\textbf{Caso de Uso} & \textbf{Descripción} \\
\midrule
CU-01: Detectar oscuridad & El sensor LDR detecta un nivel de luz bajo y el sistema enciende automáticamente el LED. \\
CU-02: Detectar presencia & El sensor ultrasónico detecta un objeto cercano y el sistema abre la puerta. \\
CU-03: Monitorizar estado & El usuario accede al dashboard web y visualiza el estado actual de luz y puerta. \\
CU-04: Forzar luz & El usuario pulsa un botón en el dashboard para encender o apagar la luz manualmente. \\
CU-05: Forzar puerta & El usuario pulsa un botón en el dashboard para abrir o cerrar la puerta manualmente. \\
\bottomrule
\end{tabular}
\end{table}

\section*{Alcance y Limitaciones}

El sistema desarrollado tiene las siguientes características y limitaciones:

\begin{itemize}
  \item \textbf{Alcance}: sistema funcional con dos sensores (luz y distancia) y dos actuadores (LED y servo), comunicación LoRa bidireccional con ACK, integración MQTT completa y dashboard web en tiempo real.
  \item \textbf{Limitaciones}: no implementa cifrado en LoRa, no tiene persistencia de datos históricos, y el dashboard no incluye autenticación de usuarios.
\end{itemize}
