En este documento se describe el desarrollo de un sistema de domótica distribuido basado en tecnologías IoT. 
El objetivo del trabajo es diseñar y poner en funcionamiento una arquitectura completa que permita, a partir de sensores físicos de 
luz y proximidad, controlar automáticamente la iluminación y la apertura de una puerta utilizando comunicaciones LoRa y un broker 
MQTT sobre una Raspberry Pi \cite{raspberrypi}.

El sistema se compone de varios nodos cooperando entre sí: un nodo de sensores dividido en maestro y esclavo, un gateway LoRa--MQTT, 
un nodo actuador y una capa de aplicaciones cliente. El enlace de inalámbrico entre los nodos físicos se realiza mediante 
LoRa, sin embargo, entre el esclavo y el maestro la comunicación es serial. Por otro lado, la integración lógica y el acceso 
desde el exterior se articula a través de topics MQTT en un broker Mosquitto.

El proyecto permite aplicar de forma práctica conceptos de comunicaciones de bajo consumo,
 diseño de protocolos ligeros, integración de servicios mediante MQTT y desarrollo de dashboards web en tiempo real. 
 Además, muestra cómo separar responsabilidades entre captura de datos, lógica de decisión, transporte y presentación.

El resto de la memoria se organiza de la siguiente forma. En el Capítulo~\ref{chap:descripcion} se presenta una descripción 
general del sistema y los objetivos perseguidos. El Capítulo~\ref{chap:diseno} detalla el diseño hardware y software adoptado, 
incluyendo el protocolo maestro--esclavo y el formato de las tramas LoRa. En el Capítulo~\ref{chap:arquitectura} se describe 
la arquitectura global y los flujos de comunicación entre nodos y el broker MQTT. El Capítulo~\ref{chap:modelodatos} resume los datos 
intercambiados y la codificación utilizada en cada capa. El Capítulo~\ref{chap:desarrollo} recoge los aspectos prácticos de 
implementación y pruebas, mientras que el Capítulo~\ref{chap:interfaces} muestra las interfaces más relevantes, como el dashboard web. 
Finalmente, en el Capítulo~\ref{chap:conclusiones} se presentan las conclusiones y posibles líneas de mejora.

