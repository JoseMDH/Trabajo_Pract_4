\documentclass[12pt,oneside,a4paper]{report}

% ====== Paquetes ======
\usepackage[spanish,es-tabla]{babel}
\usepackage[utf8]{inputenc}
\usepackage[T1]{fontenc}
\usepackage{geometry}
\geometry{margin=2.5cm}
\usepackage{graphicx}
\graphicspath{{Ilustraciones/}}
\usepackage{xcolor}
\usepackage{hyperref}
\hypersetup{
  colorlinks=true,
  linkcolor=blue,
  citecolor=blue,
  urlcolor=blue,
  pdfauthor={},
  pdftitle={Memoria de Prácticas}
}
\usepackage{caption}
\usepackage{subcaption}
\usepackage{float}
\usepackage{booktabs}
\usepackage{longtable}
\usepackage{array}
\usepackage{listings}
\usepackage{tikz}
\usetikzlibrary{positioning,arrows.meta,shapes.multipart}
\lstset{
  basicstyle=\ttfamily\small,
  breaklines=true,
  frame=single,
  numbers=left,
  numberstyle=\tiny,
  tabsize=2
}

% ====== Metadatos personalizables ======
\newcommand{\titulo}{\textbf{Sistema de Domótica con LoRa y MQTT}} % título del trabajo
\newcommand{\autores}{José Manuel Díaz Hernández \\ Nicolás Rey Alonso \\ Santiago Galindo Peralta \\ Alberto Martel Rodríguez}

% ====== Documento ======
\begin{document}

% ---- Portada ----
% ========================
%      Portada ULPGC
% ========================
\thispagestyle{empty}

\begin{titlepage}
% Ajuste para reducir el margen superior y acercar el logo al borde superior
\vspace*{4cm}

\begin{center}

% Logotipo principal EII-ULPGC (centrado y grande)
\makebox[\textwidth][c]{\includegraphics[width=1.1\textwidth]{NuevoLogoEII}}

\vspace{1.5cm}

% Título del curso (más grande que el título de la aplicación)
{\Huge \textbf{Internet de las Cosas (GII-IC)}\par}
\vspace{0.8cm}


% Título de la aplicación
{\Large \titulo \par}

\vfill

% Autores centrados como en la imagen
{\large \textbf{Autores}\par}
\vspace{3mm}
{\large
\begin{minipage}{0.7\textwidth}
\centering
\autores % cada autor separado con \\
\end{minipage}
}

\end{center}
\end{titlepage}

\clearpage

% ---- Índices ----
\pagenumbering{roman}
\setcounter{page}{1}

\tableofcontents
\clearpage

\listoffigures
\clearpage

\listoftables
\clearpage

% ---- Cuerpo ----
\pagenumbering{arabic}
\setcounter{page}{1}

\chapter{Introducción}
En este documento se describe el desarrollo de un sistema de domótica distribuido basado en tecnologías IoT. 
El objetivo del trabajo es diseñar y poner en funcionamiento una arquitectura completa que permita, a partir de sensores físicos de 
luz y proximidad, controlar automáticamente la iluminación y la apertura de una puerta utilizando comunicaciones LoRa y un broker 
MQTT sobre una Raspberry Pi \cite{raspberrypi}.

El sistema se compone de varios nodos cooperando entre sí: un nodo de sensores dividido en maestro y esclavo, un gateway LoRa--MQTT, 
un nodo actuador y una capa de aplicaciones cliente. El enlace de inalámbrico entre los nodos físicos se realiza mediante 
LoRa, sin embargo, entre el esclavo y el maestro la comunicación es serial. Por otro lado, la integración lógica y el acceso 
desde el exterior se articula a través de topics MQTT en un broker Mosquitto.

El proyecto permite aplicar de forma práctica conceptos de comunicaciones de bajo consumo,
 diseño de protocolos ligeros, integración de servicios mediante MQTT y desarrollo de dashboards web en tiempo real. 
 Además, muestra cómo separar responsabilidades entre captura de datos, lógica de decisión, transporte y presentación.

El resto de la memoria se organiza de la siguiente forma. En el Capítulo~\ref{chap:descripcion} se presenta una descripción 
general del sistema y los objetivos perseguidos. El Capítulo~\ref{chap:diseno} detalla el diseño hardware y software adoptado, 
incluyendo el protocolo maestro--esclavo y el formato de las tramas LoRa. En el Capítulo~\ref{chap:arquitectura} se describe 
la arquitectura global y los flujos de comunicación entre nodos y el broker MQTT. El Capítulo~\ref{chap:modelodatos} resume los datos 
intercambiados y la codificación utilizada en cada capa. El Capítulo~\ref{chap:desarrollo} recoge los aspectos prácticos de 
implementación y pruebas, mientras que el Capítulo~\ref{chap:interfaces} muestra las interfaces más relevantes, como el dashboard web. 
Finalmente, en el Capítulo~\ref{chap:conclusiones} se presentan las conclusiones y posibles líneas de mejora.



\chapter{Descripción de la aplicación y objetivos}\label{chap:descripcion}
% Descripción de la aplicación incluyendo objetivos

En este trabajo se ha desarrollado un sistema de domótica distribuido orientado a un entorno doméstico sencillo: una puerta de acceso y un punto de iluminación. El sistema toma decisiones automáticamente a partir de sensores de distancia (ultrasonidos) y un sensor de luz (LDR), y además permite forzar manualmente los estados desde aplicaciones externas vía MQTT (por ejemplo, un dashboard web o Node-RED).

El público objetivo es principalmente docente: estudiantes de la asignatura de Internet de las Cosas que necesiten un ejemplo completo de arquitectura IoT (sensores, pasarela, protocolo de campo, broker MQTT y aplicaciones cliente). No obstante, la solución es extensible a escenarios reales de monitorización y control de acceso en pequeña escala.

\section*{Objetivos}

Los objetivos principales del proyecto son:

\begin{itemize}
  \item Diseñar y desplegar una arquitectura IoT completa que conecte sensores físicos, un nodo actuador y una pasarela basada en Raspberry Pi usando LoRa como red de campo y MQTT como capa de integración.
  \item Implementar un protocolo maestro--esclavo ligero para la lectura y configuración de sensores (ultrasonidos y LDR) y un formato de tramas LoRa con direcciones, identificadores de mensaje y ACK de aplicación.
  \item Integrar un broker MQTT (Mosquitto) y un puente serie--MQTT que permita exponer los datos de los sensores y controlar el actuador desde aplicaciones externas de forma desacoplada.
  \item Desarrollar una interfaz web sencilla que visualice en tiempo real el estado de luz y puerta, y que permita enviar comandos manuales reutilizando la infraestructura MQTT existente.
\end{itemize}

Desde el punto de vista funcional, el sistema ofrece:

\begin{itemize}
  \item Encendido y apagado automático de la luz en función del nivel medido por la LDR (umbrales configurados en el maestro).
  \item Apertura y cierre de una puerta simulada mediante un servo, a partir de la detección de presencia por dos sensores ultrasónicos.
  \item Publicación de los estados lógicos (luz y puerta) en topics MQTT para monitorización y análisis.
  \item Posibilidad de forzar manualmente los estados desde la red MQTT sin necesidad de acceder físicamente a los nodos LoRa.
\end{itemize}


\chapter{Diseño}\label{chap:diseno}
% Diseño

En esta sección se describen las principales decisiones de diseño tanto a nivel lógico como de interacción con el sistema.

\section*{Patrones y decisiones}

A nivel lógico se ha optado por una arquitectura por capas:

\begin{itemize}
  \item \textbf{Capa de sensores}: formada por un esclavo y un maestro (MKR WAN 1310). El esclavo encapsula el 
  acceso hardware a los sensores (I\textsuperscript{2}C para SRF01/SRF02 y analógico para la LDR) detrás de un protocolo serie 
  simple. El maestro actúa como único punto que interpreta las medidas, aplica umbrales y decide estados lógicos.
  \item \textbf{Capa de transporte LoRa}: el maestro y el gateway intercambian tramas LoRa con cabeceras explícitas 
  (direcciones, identificador de mensaje, longitud), lo que permite dirigir mensajes a distintos nodos (gateway, actuador) y 
  asociar ACKs a comandos concretos.
  \item \textbf{Capa de integración MQTT}: en la Raspberry Pi, un proceso puente traduce entre el protocolo serie del gateway y
   mensajes MQTT, mapeando topics internos (\texttt{sensor/0}, \texttt{sensor/1}) a topics lógicos (\texttt{sensores/puerta}, 
   \texttt{sensores/luz}). Esta capa aplica además políticas de \emph{rate limiting} y eliminación de duplicados para no saturar LoRa.
  \item \textbf{Capa de presentación}: formada por el dashboard web y 
  herramientas como \texttt{mosquitto\_sub}/\texttt{mosquitto\_pub}. Estas aplicaciones solo ven topics MQTT y no necesitan conocer 
  detalles de LoRa ni del protocolo maestro--esclavo.
\end{itemize}

Esta separación responde a dos decisiones clave:

\begin{itemize}
  \item Mantener los nodos embebidos (maestro, esclavo, actuador) lo más simples posible, delegando el tratamiento de mensajes, 
  la conversión de formatos y la integración con otras aplicaciones en la Raspberry Pi.
  \item Utilizar formatos binarios compactos en los enlaces de menor ancho de banda (LoRa y UART) y formatos más ricos 
  (cadenas y JSON) en MQTT, donde el coste en bytes es menos crítico.
\end{itemize}

En cuanto al control, el maestro solo envía nuevas órdenes cuando detecta un cambio de estado (por ejemplo, de luz=0 a luz=1), 
y el actuador siempre responde con un ACK con el mismo identificador de mensaje. El gateway reintenta automáticamente las 
transmisiones si no recibe ese ACK dentro de un tiempo máximo.

\section*{Prototipos/Mockups}

Aunque el proyecto no es una aplicación móvil, se ha diseñado una interfaz web sencilla como dashboard de usuario final. 
La página muestra dos tarjetas principales:

\begin{itemize}
  \item \textbf{Sensor de luz}: icono y estado textual (\emph{oscuro} / \emph{iluminado}), con dos botones para enviar manualmente 
  valores 0 o 1 sobre el topic \texttt{sensores/luz}.
  \item \textbf{Sensor de puerta}: icono de puerta y estado (\emph{cerrada/nadie} o \emph{abierta/gente}), de nuevo con botones 
  para forzar los estados 0 o 1 sobre \texttt{sensores/puerta}.
\end{itemize}

Además, el dashboard incluye indicadores de conexión (estado de la conexión web y del broker MQTT) y un panel de log donde 
se registran los eventos relevantes (nuevos valores recibidos, comandos enviados, pérdidas de conexión).

A nivel de diseño de interacción se ha priorizado:

\begin{itemize}
  \item Representar los estados con emojis e indicadores de color para facilitar una lectura rápida.
  \item Mostrar claramente si el sistema está conectado al broker MQTT y si la web está sincronizada con el estado actual.
  \item Registrar un histórico de eventos para ayudar en la depuración durante las pruebas.
\end{itemize}


\begin{figure}[H]
  \centering
  \begin{subfigure}[b]{0.48\textwidth}
    \centering
    \includegraphics[width=\linewidth]{./CapturasDashboard/1.png}
    \caption{Dashboard — Escritorio (1)}
    \label{fig:dash-escritorio-1}
  \end{subfigure}%
  \hfill
  \begin{subfigure}[b]{0.48\textwidth}
    \centering
    \includegraphics[width=\linewidth]{./CapturasDashboard/2.png}
    \caption{Dashboard — Escritorio (2)}
    \label{fig:dash-escritorio-2}
  \end{subfigure}
  \vspace{4mm}
  \begin{subfigure}[b]{0.48\textwidth}
    \centering
    \includegraphics[width=0.48\linewidth]{./CapturasDashboard/3.png}
    \caption{Dashboard — Móvil (3)}
    \label{fig:dash-movil-3}
  \end{subfigure}%
  \hfill
  \begin{subfigure}[b]{0.48\textwidth}
    \centering
    \includegraphics[width=0.48\linewidth]{./CapturasDashboard/4.png}
    \caption{Dashboard — Móvil (4)}
    \label{fig:dash-movil-4}
  \end{subfigure}
  \caption{Capturas del dashboard en escritorio (1–2) y móvil (3–4).}
  \label{fig:capturas-dashboard}
\end{figure}

\subsection*{Diseño de la arquitectura del sistema}
La arquitectura del sistema se basa en una comunicación jerárquica y modular entre los distintos componentes, cada uno con 
responsabilidades claras:
\begin{itemize}
  \item \textbf{Esclavo de sensores}: encargado de la adquisición de datos desde los sensores físicos (LDR y SRF01/SRF02) y de 
  enviarlos al maestro a través del canal serie.
  \item \textbf{Maestro de sensores}: recibe los datos del esclavo, procesa las mediciones aplicando umbrales para determinar 
  estados lógicos, y gestiona la comunicación inalámbrica mediante LoRa con el gateway.
  \item \textbf{Gateway LoRa}: actúa como intermediario entre el maestro y la Raspberry Pi, recibiendo los mensajes LoRa y 
  retransmitiéndolos a través del puerto serie y recibiendo los mensajes MQTT desde la raspberry y retransmitiéndolos por LoRa.
  \item \textbf{Raspberry Pi}: ejecuta un proceso puente que traduce los mensajes del gateway a formato MQTT, aplicando 
  políticas de filtrado y mapeo de topics. Además, aloja el broker MQTT y el dashboard web.
  \item \textbf{Dashboard web}: proporciona una interfaz de usuario para visualizar el estado de los sensores y enviar comandos
   manuales a través de MQTT.
\end{itemize}
\subsubsection*{Diagrama de arquitectura del sistema}
\begin{figure}[H]
  \centering
  \shorthandoff{<>}%
  \resizebox{\textwidth}{!}{%
    \begin{tikzpicture}[
      box/.style={draw, rounded corners, minimum width=30mm, minimum height=10mm, align=center, fill=white, font=\scriptsize},
      serial/.style={-{Stealth}, thick},
      lora/.style={-{Stealth}, thick, dashed},
      lbl/.style={font=\tiny, align=center}
    ]

    % Nodes
    \node[box] (esclavo) {ESCLAVO\\(Arduino UNO)\\Luz/SRF};
    \node[box,right=30mm of esclavo] (maestro) {MAESTRO\\(MKR WAN)\\LoRa TX};
    \node[box,below=15mm of maestro] (gateway) {GATEWAY\\(MKR WAN)\\LoRa RX/TX};
    \node[box,left=30mm of gateway] (raspi) {RASPBERRY PI\\MQTT Broker};
    \node[box,right=30mm of gateway] (actuador) {ACTUADOR\\(Arduino UNO)\\LED/Servo};
    \node[box,below=15mm of raspi] (apps) {APLICACIONES\\Dashboard};

    % Connections
    \draw[serial] (esclavo) -- node[midway,above,lbl]{Serial} (maestro);
    \draw[lora] (maestro.south) .. controls +(0,-8mm) and +(0,8mm) .. node[midway,right=1mm,lbl]{LoRa} (gateway.north);
    \draw[<->, thick] (gateway.west) -- node[midway,above,lbl]{Serial (GPIO UART)} (raspi.east);
    \draw[lora] (gateway.east) -- node[midway,above,lbl]{LoRa} (actuador.west);
    \draw[serial] (raspi) -- node[midway,left,lbl]{MQTT} (apps);

    \end{tikzpicture}%
  }%
  \shorthandon{<>}%

  % Leyenda separada, debajo del dibujo
  \vspace{2mm}
  \begin{tikzpicture}[
    serial/.style={-{Stealth}, thick},
    lora/.style={-{Stealth}, thick, dashed}
  ]
    \node[anchor=west,font=\small] at (0,0) {\textbf{Leyenda:}};
    \draw[serial] (0,-0.5) -- ++(1,0) node[right=2mm,font=\tiny] {Serial / MQTT};
    \draw[lora] (4,-0.5) -- ++(1,0) node[right=2mm,font=\tiny] {LoRa (dashed)};
  \end{tikzpicture}

  \caption{Diagrama de arquitectura del sistema. Líneas continuas: comunicación serie/MQTT. Líneas discontinuas: LoRa inalámbrico.}
  \label{fig:diagrama-arquitectura}
\end{figure}

\subsection*{Diseño de circuitos con KiCad}
Para homogeneizar el diseño de los circuitos entre todos los miembros del grupo, se utilizó KiCad como herramienta común
 para el esquemático y el diseño de PCB. En KiCad se modeló el divisor de tensión de la LDR, lo que permitió compartir los mismos 
 ficheros de proyecto y reducir errores de cableado entre montajes físicos.

KiCad es una suite de diseño electrónico libre y de código abierto, ampliamente utilizada en entornos académicos y profesionales 
para el diseño de circuitos impresos \cite{kicad}. Además, su documentación oficial facilita la configuración de reglas de 
diseño y la generación de los \emph{layouts} necesarios para el montaje \cite{kicad-docs}.


\chapter{Arquitectura}\label{chap:arquitectura}
% Arquitectura

En este capítulo se describe la arquitectura global del sistema, los flujos de comunicación entre los distintos nodos y los protocolos 
utilizados en cada capa.

\section*{Visión General}

El sistema sigue una arquitectura distribuida de tipo \emph{sensor-gateway-actuador} con integración en la nube a través de MQTT. 
Los componentes principales son:

\begin{enumerate}
  \item \textbf{Nodo de sensores}: compuesto por un esclavo (Arduino MKR WAN 1310) y un maestro (MKR WAN 1310).
  \item \textbf{Gateway LoRa}: Arduino MKR WAN 1310 conectado a la Raspberry Pi.
  \item \textbf{Raspberry Pi}: ejecuta el broker MQTT y el puente serie-MQTT.
  \item \textbf{2 Fuentes de alimentación}: proporcionan energía a los sensores y dispositivos.
  \item \textbf{Nodo actuador}: Arduino MKR WAN 1310 conectado a los motores.
  \item \textbf{Aplicaciones cliente}: dashboard web.
\end{enumerate}

\section*{Diagrama de Flujo de Datos}

El flujo de datos desde los sensores hasta las aplicaciones cliente sigue el siguiente recorrido:

\begin{figure}[H]
  \centering
  \begin{tikzpicture}[
    box/.style={draw, rounded corners, minimum width=28mm, minimum height=10mm,
                align=center, fill=blue!10},
    arrow/.style={-{Stealth}, thick},
    node distance=18mm
  ]
    % Fila superior
    \node[box] (sensor)  {Sensores\\(LDR, SRF)};
    \node[box, right=30mm of sensor] (esclavo) {Esclavo};
    \node[box, right=30mm of esclavo] (maestro) {Maestro};

    % Fila inferior (más ancha, hacia la izquierda)
    \node[box, below=25mm of esclavo, xshift=15mm] (gateway) {Gateway};
    \node[box, left=35mm of gateway] (raspi) {Raspberry Pi};

    % Flechas fila superior
    \draw[arrow] (sensor) -- node[above, font=\scriptsize]{Analógico/I2C} (esclavo);
    \draw[arrow] (esclavo) -- node[above, font=\scriptsize]{Serial} (maestro);

    % Flecha Maestro -> Gateway (diagonal hacia abajo-izquierda)
    \draw[arrow, dashed] (maestro.south) -- node[sloped, above, font=\scriptsize]{LoRa}
        (gateway.north east);

    % Flecha Gateway -> Raspberry (horizontal hacia la izquierda)
    \draw[arrow] (gateway.west) -- node[above, font=\scriptsize]{Serial (GPIO UART)}
        (raspi.east);
  \end{tikzpicture}
  \caption{Flujo de datos desde sensores, pasando por el maestro, hasta la Raspberry Pi.}
  \label{fig:flujo-sensores}
\end{figure}


\section*{Protocolo Maestro-Esclavo (Serial)}

La comunicación entre el esclavo y el maestro utiliza un protocolo serie sencillo basado en tramas estructuradas. 
El esclavo envía periódicamente las lecturas de los sensores al maestro.

\begin{table}[H]
\centering
\caption{Estructura de respuesta del esclavo al maestro.}
\label{tab:protocolo-esclavo}
\begin{tabular}{lll}
\toprule
\textbf{Campo} & \textbf{Tamaño} & \textbf{Descripción} \\
\midrule
Código respuesta & 1 byte & Tipo de dato (distancia, luz, error) \\
ID Sensor & 1 byte & Identificador del sensor \\
Dato (MSB) & 1 byte & Byte alto de la medida \\
Dato (LSB) & 1 byte & Byte bajo de la medida \\
\bottomrule
\end{tabular}
\end{table}

\section*{Protocolo LoRa}

Las comunicaciones LoRa entre el maestro, el gateway y el actuador utilizan un formato de paquete común con cabecera explícita:

\begin{table}[H]
\centering
\caption{Formato de paquete LoRa.}
\label{tab:formato-lora}
\begin{tabular}{llp{6cm}}
\toprule
\textbf{Campo} & \textbf{Tamaño} & \textbf{Descripción} \\
\midrule
Destino & 1 byte & Dirección del nodo destino \\
Origen & 1 byte & Dirección del nodo emisor \\
Msg ID (MSB) & 1 byte & Identificador de mensaje (byte alto) \\
Msg ID (LSB) & 1 byte & Identificador de mensaje (byte bajo) \\
Longitud & 1 byte & Tamaño del payload en bytes \\
Payload & N bytes & Datos del mensaje \\
\bottomrule
\end{tabular}
\end{table}

Las direcciones asignadas a cada nodo son:

\begin{table}[H]
\centering
\caption{Direcciones LoRa de los nodos del sistema.}
\label{tab:direcciones-lora}
\begin{tabular}{ll}
\toprule
\textbf{Nodo} & \textbf{Dirección} \\
\midrule
Maestro (sensores) & \texttt{0x04} \\
Gateway & \texttt{0x05} \\
Actuador & \texttt{0x06} \\
Broadcast & \texttt{0xFF} \\
\bottomrule
\end{tabular}
\end{table}

\section*{Protocolo Serial Gateway-Raspberry}

El gateway y la Raspberry Pi se comunican mediante un protocolo binario delimitado por caracteres STX/ETX:

\begin{table}[H]
\centering
\caption{Formato de trama serial entre Gateway y Raspberry Pi.}
\label{tab:protocolo-serial-gateway}
\begin{tabular}{llp{5cm}}
\toprule
\textbf{Campo} & \textbf{Valor/Tamaño} & \textbf{Descripción} \\
\midrule
STX & \texttt{0x02} & Inicio de trama \\
Tipo & 1 byte & R=RX, T=TX, A=ACK, N=NACK, S=Status \\
Topic Len & 1 byte & Longitud del topic \\
Topic & N bytes & Nombre del topic \\
Payload Len & 1 byte & Longitud del payload \\
Payload & N bytes & Datos \\
ETX & \texttt{0x03} & Fin de trama \\
\bottomrule
\end{tabular}
\end{table}

\section*{Integración MQTT}

El broker MQTT (Mosquitto) en la Raspberry Pi expone los siguientes topics:

\begin{table}[H]
\centering
\caption{Topics MQTT del sistema.}
\label{tab:topics-mqtt}
\begin{tabular}{llp{5cm}}
\toprule
\textbf{Topic} & \textbf{Dirección} & \textbf{Descripción} \\
\midrule
\texttt{sensores/luz} & Publicación & Estado del sensor de luz (0/1) \\
\texttt{sensores/puerta} & Publicación & Estado del sensor de proximidad (0/1) \\
\texttt{lora/rx} & Publicación & Mensajes LoRa en bruto (JSON) \\
\texttt{lora/tx} & Suscripción & Enviar mensaje LoRa genérico \\
\texttt{actuador/comando} & Suscripción & Comandos para el actuador \\
\bottomrule
\end{tabular}
\end{table}

El puente MQTT-LoRa (\texttt{mqtt\_lora\_bridge.py}) realiza el mapeo entre los topics internos de LoRa y los topics MQTT legibles:

\begin{itemize}
  \item \texttt{sensor/0} $\rightarrow$ \texttt{sensores/puerta}
  \item \texttt{sensor/1} $\rightarrow$ \texttt{sensores/luz}
\end{itemize}

\section*{Flujo de Control (Actuador)}

Cuando una aplicación cliente desea controlar el actuador, el flujo es el siguiente:

\begin{enumerate}
  \item La aplicación publica en \texttt{sensores/luz} o \texttt{sensores/puerta}.
  \item El bridge recibe el mensaje MQTT y lo convierte a trama serial.
  \item El gateway recibe la trama y la transmite por LoRa al actuador.
  \item El actuador responde con ACK y ejecuta la acción.
  \item El gateway recibe el ACK y notifica al bridge.
  \item Si no hay ACK en el tiempo límite, el gateway reintenta (hasta 3 veces).
\end{enumerate}

\section*{Políticas de Calidad de Servicio}

Para garantizar la fiabilidad del sistema se implementan las siguientes políticas:

\begin{itemize}
  \item \textbf{Rate limiting}: el bridge limita los envíos a un mínimo de 150~ms entre mensajes para no saturar LoRa.
  \item \textbf{Eliminación de duplicados}: mensajes idénticos recibidos en una ventana de 2 segundos se ignoran.
  \item \textbf{Reintentos con ACK}: el gateway reintenta hasta 3 veces si no recibe confirmación del actuador.
  \item \textbf{Timeout configurable}: el tiempo máximo de espera de ACK es de 2 segundos, ajustado para SF10 y BW 62.5~kHz.
\end{itemize}

\section*{Diagrama de Secuencia}

A continuación se muestra el diagrama de secuencia para un ciclo completo de detección y actuación:

\begin{figure}[H]
  \centering
  \begin{tikzpicture}[
    node distance=18mm,
    actor/.style={draw, minimum width=18mm, minimum height=8mm, align=center},
    arrow/.style={-{Stealth}, thick},
    darrow/.style={-{Stealth}, thick, dashed}
  ]
    % Actores
    \node[actor] (esc) {Esclavo};
    \node[actor, right=of esc] (mae) {Maestro};
    \node[actor, right=of mae] (gw) {Gateway};
    \node[actor, right=of gw] (rpi) {Raspberry};
    \node[actor, right=of rpi] (act) {Actuador};
    
    % Líneas de vida
    \foreach \n in {esc, mae, gw, rpi, act} {
      \draw[dashed, gray] (\n.south) -- ++(0,-55mm);
    }
    
    % Mensajes
    \draw[arrow] ([yshift=-10mm]esc.south) -- node[above, font=\scriptsize]{Lectura sensor} ([yshift=-10mm]mae.south);
    \draw[darrow] ([yshift=-18mm]mae.south) -- node[above, font=\scriptsize]{LoRa: sensor/1} ([yshift=-18mm]gw.south);
    \draw[arrow] ([yshift=-26mm]gw.south) -- node[above, font=\scriptsize]{Serial} ([yshift=-26mm]rpi.south);
    \draw[arrow] ([yshift=-34mm]rpi.south) -- node[above, font=\scriptsize]{Serial: cmd} ([yshift=-34mm]gw.south);
    \draw[darrow] ([yshift=-42mm]gw.south) -- node[above, font=\scriptsize]{LoRa: cmd} ([yshift=-42mm]act.south);
    \draw[darrow] ([yshift=-50mm]act.south) -- node[above, font=\scriptsize]{LoRa: ACK} ([yshift=-50mm]gw.south);
  \end{tikzpicture}
  \caption{Diagrama de secuencia: detección de luz y actuación.}
  \label{fig:secuencia-luz}
\end{figure}

\section*{Consideraciones de Seguridad}

Aunque el sistema no implementa cifrado en las comunicaciones LoRa (fuera del alcance de este proyecto académico), 
se aplican medidas básicas:

\begin{itemize}
  \item Filtrado por dirección de origen en el actuador.
  \item Sync Word compartido (\texttt{0x12}) que actúa como identificador de red.
  \item Comunicación MQTT local (localhost) sin exposición a Internet.
\end{itemize}

En un despliegue real se debería añadir cifrado en la capa de aplicación y autenticación MQTT con usuario/contraseña o certificados.


\chapter{Modelo de Datos}\label{chap:modelodatos}
% Modelo de Datos - Sensores

En este capítulo se describe el modelo de datos utilizado en el sistema, incluyendo la estructura de configuración de los sensores, 
los formatos de mensajes intercambiados y la codificación utilizada en cada capa.

\section*{Notas Técnicas}

\begin{itemize}
  \item Todas las placas conectadas a la antena LoRa están conectadas a una batería para manejar la tensión y no quemar la placa.
  \item Se heredó código de las prácticas 2 y 3 de la asignatura.
\end{itemize}

\section*{Sensores Utilizados}

El sistema utiliza tres sensores físicos para la adquisición de datos del entorno:

\begin{table}[H]
\centering
\caption{Sensores del sistema.}
\label{tab:sensores}
\begin{tabular}{llll}
\toprule
\textbf{Sensor} & \textbf{Tipo} & \textbf{Interfaz} & \textbf{Función} \\
\midrule
SRF01 & Ultrasónico & I\textsuperscript{2}C & Detección de distancia \\
SRF02 & Ultrasónico & I\textsuperscript{2}C & Detección de distancia \\
LDR & Fotorresistencia & Analógico (A1) & Nivel de luz ambiental \\
\bottomrule
\end{tabular}
\end{table}

\section*{Esclavo (\texttt{Esclavo.ino})}

El esclavo se encarga de la gestión directa de sensores. Cada sensor dispone de una estructura de configuración propia que almacena 
su dirección, unidad de medida, retardo mínimo, modo de funcionamiento (periódico o puntual), periodo de muestreo y último valor leído.

\clearpage

\begin{lstlisting}[language=C++, caption={Estructura de configuración de sensores.}]
struct SensorConfig {
  uint8_t address;      // Direccion I2C del sensor
  uint8_t unit;         // Unidad de medida (cm, inch, us)
  uint16_t delayMs;     // Retardo minimo entre lecturas
  bool periodic;        // Modo periodico activo
  uint16_t periodMs;    // Periodo de muestreo (ms)
  unsigned long lastShot; // Timestamp ultima lectura
  uint16_t lastMeasure; // Ultimo valor leido
  bool active;          // Sensor detectado/activo
  char name[8];         // Nombre del sensor
};
\end{lstlisting}

Durante la fase de inicialización, el dispositivo detecta automáticamente la presencia de los sensores I\textsuperscript{2}C y 
habilita por defecto las mediciones periódicas (1~segundo) para aquellos sensores activos.

\subsection*{Funciones de Lectura}

Las lecturas de los sensores se realizan mediante las siguientes funciones:

\begin{lstlisting}[language=C++, caption={Funciones de lectura de sensores.}]
// Lectura de sensores ultrasonicos SRF02
uint16_t readSRF02(uint8_t address, uint8_t unit) {
  Wire.beginTransmission(address);
  Wire.write(0x00);      // Registro de comando
  Wire.write(unit);      // 0x51=cm, 0x50=inch, 0x52=us
  Wire.endTransmission();
  delay(70);             // Tiempo de medicion
  Wire.requestFrom(address, 2);
  uint16_t distance = (Wire.read() << 8) | Wire.read();
  return distance;
}

// Lectura del fotorresistor
uint16_t readLDR() {
  return analogRead(A1);  // Valor 0-1023
}
\end{lstlisting}

\clearpage

\subsection*{Protocolo de Respuesta}

El esclavo envía las lecturas al maestro mediante un protocolo estructurado:

\begin{lstlisting}[language=C++, caption={Envío de respuesta al maestro.}]
void sendResponse(uint8_t code, uint8_t* data, uint8_t len) {
  Serial1.write(RESP_START);   // Marcador inicio
  Serial1.write(code);         // Codigo de respuesta
  Serial1.write(len);          // Longitud de datos
  if (len > 0 && data != nullptr) {
    Serial1.write(data, len);
  }
  Serial1.write(RESP_END);     // Marcador fin
}
\end{lstlisting}

\section*{Maestro (\texttt{Maestro.ino})}

El maestro supervisa el sistema de sensores, interpreta las mediciones recibidas y las publica como pares etiqueta-valor a 
través de la red LoRa.

\subsection*{Umbrales de Decisión}

Las medidas crudas recibidas del esclavo son procesadas aplicando umbrales para convertirlas en estados binarios:

\begin{lstlisting}[language=C++, caption={Umbrales de decisión.}]
#define LIGHT_THRESHOLD   500   // luz < 500 -> oscuro (1)
#define DISTANCE_THRESHOLD 100  // distancia < 100cm -> cerca (1)
\end{lstlisting}

\begin{table}[H]
\centering
\caption{Interpretación de valores de sensores.}
\label{tab:umbrales}
\begin{tabular}{llll}
\toprule
\textbf{Sensor} & \textbf{Condición} & \textbf{Estado} & \textbf{Acción} \\
\midrule
LDR & valor < 500 & 1 (oscuro) & Encender LED \\
LDR & valor $\geq$ 500 & 0 (iluminado) & Apagar LED \\
SRF & distancia < 100~cm & 1 (cerca) & Abrir puerta \\
SRF & distancia $\geq$ 100~cm & 0 (lejos) & Cerrar puerta \\
\bottomrule
\end{tabular}
\end{table}

\clearpage

\subsection*{Estructura de Mensajes}

Para la comunicación LoRa, el maestro utiliza una estructura que mantiene el estado de cada sensor:

\begin{lstlisting}[language=C++, caption={Estructura de mensaje de sensor.}]
struct SensorMessage {
  uint8_t payload;        // Valor a enviar (0 o 1)
  bool pending;           // Hay mensaje pendiente
  uint8_t lastSentValue;  // Ultimo valor enviado
};

// Mensajes pendientes para cada sensor
SensorMessage sensorMessages[2];  // [0]=puerta, [1]=luz
\end{lstlisting}

\subsection*{Configuración LoRa}

El maestro configura el módulo LoRa integrado en el MKR WAN 1310 con los siguientes parámetros:

\begin{lstlisting}[language=C++, caption={Configuración LoRa del maestro.}]
#define LORA_LOCAL_ADDRESS      0x04  // Direccion del maestro
#define LORA_GATEWAY_ADDRESS    0x05  // Direccion del gateway
#define LORA_FREQUENCY          868E6 // 868 MHz (Europa)
#define LORA_BW                 62.5E3 // Ancho de banda
#define LORA_SF                 10    // Spreading Factor
#define LORA_CR                 5     // Coding Rate 4/5
#define LORA_TP                 2     // Potencia TX (dBm)
#define LORA_SYNC_WORD          0x12  // Palabra de sincronizacion
#define LORA_PREAMBLE_LENGTH    8     // Longitud preambulo
\end{lstlisting}

Decidimos usar un ancho de banda de 62.5~kHz y un Spreading Factor de 10 para equilibrar
alcance y tasa de datos, adecuados para un entorno urbano.

\subsection*{Formato del Payload LoRa}

El payload enviado por el maestro al gateway tiene el siguiente formato:

\begin{table}[H]
\centering
\caption{Formato del payload de sensores.}
\label{tab:payload-sensor}
\begin{tabular}{llp{5cm}}
\toprule
\textbf{Campo} & \textbf{Tamaño} & \textbf{Descripción} \\
\midrule
TopicLen & 1 byte & Longitud del nombre del topic \\
Topic & N bytes & Nombre del topic (ej: ``sensor/1'') \\
Valor & 1 byte & Estado binario del sensor (0 o 1) \\
\bottomrule
\end{tabular}
\end{table}

\clearpage

\subsection*{Política de Envío}

El sistema de envío está diseñado para ser reactivo y eficiente:

\begin{itemize}
  \item Solo se envía un mensaje cuando hay un \textbf{cambio de estado} (de 0 a 1 o viceversa).
  \item Se mantiene un \textbf{intervalo mínimo} de 500~ms entre envíos para no saturar el canal.
  \item Los envíos se \textbf{alternan entre sensores} para distribuir la carga.
\end{itemize}

\begin{lstlisting}[language=C++, caption={Control de envío LoRa.}]
const unsigned long LORA_SEND_INTERVAL = 500;  // 500ms minimo
volatile bool sendDistanceNext = false;        // Alternar sensores
\end{lstlisting}

\section*{Resumen del Modelo de Datos}

\begin{table}[H]
\centering
\caption{Resumen de topics y valores.}
\label{tab:resumen-datos}
\begin{tabular}{lllp{4cm}}
\toprule
\textbf{Topic Interno} & \textbf{Topic MQTT} & \textbf{Valores} & \textbf{Significado} \\
\midrule
sensor/0 & sensores/puerta & 0, 1 & 0=nadie, 1=presencia \\
sensor/1 & sensores/luz & 0, 1 & 0=iluminado, 1=oscuro \\
\bottomrule
\end{tabular}
\end{table}


\chapter{Desarrollo}\label{chap:desarrollo}
% Desarrollo - Actuador

En este capítulo se describe el desarrollo del nodo actuador, encargado de ejecutar las acciones físicas en respuesta a los estados detectados por los sensores. El actuador recibe comandos a través de LoRa desde el gateway y controla dos elementos: un LED para la iluminación y un servomotor para la apertura y cierre de una puerta.

\section*{Hardware del Actuador}

El nodo actuador está compuesto por los siguientes elementos:

\begin{itemize}
  \item \textbf{Arduino UNO}: microcontrolador que ejerce de fuente.
  \item \textbf{Arduino mkr WAN 1310}: microcontrolador principal que ejecuta la lógica de control y la comunicación LoRa.
  \item \textbf{LED}: indicador luminoso conectado al pin 6, representa la iluminación del entorno.
  \item \textbf{Servo SG90}: servomotor conectado al pin 7, simula el mecanismo de apertura de una puerta.
\end{itemize}

\section*{Configuración LoRa}

El actuador debe utilizar exactamente la misma configuración LoRa que el resto de nodos del sistema para garantizar la interoperabilidad:

\begin{lstlisting}[language=C++, caption={Configuración LoRa del actuador.}]
LoRaConfig_t nodeConfig = {6, 10, 5, 2};
// BW = 62.5 kHz (indice 6)
// SF = 10 (Spreading Factor)
// CR = 4/5 (Coding Rate)
// TxPwr = 2 dBm
\end{lstlisting}

La dirección LoRa asignada al actuador es \texttt{0x06}, mientras que solo acepta mensajes provenientes del gateway 
(\texttt{0x05}). Esto proporciona una capa básica de seguridad al filtrar emisores no autorizados.

\section*{Protocolo de Comandos}

Los comandos recibidos por el actuador siguen un formato binario compacto de dos bytes:

\begin{table}[H]
\centering
\caption{Formato del payload de comandos para el actuador.}
\label{tab:formato-comando-actuador}
\begin{tabular}{lll}
\toprule
\textbf{Byte} & \textbf{Campo} & \textbf{Descripción} \\
\midrule
0 & Tipo & 0 = Luz, 1 = Puerta \\
1 & Valor & Estado a aplicar \\
\bottomrule
\end{tabular}
\end{table}

Para el control de luz:
\begin{itemize}
  \item Valor 0: Apagar LED.
  \item Valor 1: Encender LED.
\end{itemize}

Para el control de puerta:
\begin{itemize}
  \item Valor 0: Cerrar puerta (servo a 10°).
  \item Valor 1, 2 o 3: Abrir puerta (servo a 160°).
\end{itemize}

\section*{Sistema de ACK}

El actuador implementa un sistema de confirmación (ACK) para garantizar la entrega fiable de comandos. Cada vez que recibe un paquete válido, responde inmediatamente con un ACK que incluye:

\begin{itemize}
  \item Dirección de destino (el gateway).
  \item Dirección de origen (el actuador).
  \item Identificador de mensaje (el mismo que el comando recibido).
  \item Marcador de ACK (\texttt{0xAC}).
  \item Estado (0 = éxito, 1 = error).
\end{itemize}

\begin{lstlisting}[language=C++, caption={Función de envío de ACK.}]
void sendAck(uint8_t dest, uint16_t msgId, uint8_t status) {
  LoRa.beginPacket();
  LoRa.write(dest);
  LoRa.write(localAddress);
  LoRa.write((uint8_t)(msgId >> 8));
  LoRa.write((uint8_t)(msgId & 0xFF));
  LoRa.write((uint8_t)2);
  LoRa.write(ACK_MARKER);
  LoRa.write(status);
  LoRa.endPacket();
}
\end{lstlisting}

\section*{Control de Duplicados}

Para evitar que un mismo comando se ejecute múltiples veces (por ejemplo, debido a retransmisiones del gateway), el actuador mantiene
 el identificador del último mensaje procesado. Si recibe un paquete con el mismo \texttt{msgId}, responde con ACK pero no vuelve a 
 ejecutar la acción:

\begin{lstlisting}[language=C++, caption={Detección de mensajes duplicados.}]
uint16_t lastProcessedMsgId = 0xFFFF;

// En el bucle de recepcion:
bool isDuplicate = (msgId == lastProcessedMsgId);
if (!isDuplicate) {
  // Procesar comando
  lastProcessedMsgId = msgId;
}
// SIEMPRE enviar ACK
sendAck(sender, msgId, ackStatus);
\end{lstlisting}

\section*{Funciones de Control}

Las funciones que aplican los estados físicos son directas y robustas:

\begin{lstlisting}[language=C++, caption={Control del LED y servo.}]
void aplicarLuz(uint8_t v) {
  if (v == 1) {
    digitalWrite(pinLed, HIGH);
  } else if (v == 0) {
    digitalWrite(pinLed, LOW);
  }
}

void aplicarPuerta(uint8_t v) {
  if (v == 0) {
    posicion = 10;
    miServo.write(posicion);
  } else if (v == 1 || v == 2 || v == 3) {
    posicion = 160;
    miServo.write(posicion);
  }
}
\end{lstlisting}

\section*{Flujo de Operación}

El actuador opera en modo \emph{polling}, verificando continuamente si hay paquetes LoRa disponibles:

\begin{enumerate}
  \item Inicialización del hardware (servo, LED, LoRa).
  \item Bucle principal: verificar si hay paquete LoRa disponible.
  \item Si hay paquete: leer cabecera y verificar destinatario.
  \item Verificar que el emisor sea el gateway autorizado.
  \item Extraer tipo y valor del payload.
  \item Si no es duplicado: ejecutar acción correspondiente.
  \item Enviar ACK al gateway.
  \item Volver al paso 2.
\end{enumerate}

Este diseño garantiza que el actuador responda rápidamente a los comandos mientras mantiene un consumo de recursos bajo, adecuado para un microcontrolador con recursos limitados como el Arduino UNO.


\chapter{Interfaces}\label{chap:interfaces}
% Interfaces - Broker, Dashboard y Gateway

En este capítulo se describen las interfaces de software del sistema: el gateway LoRa, 
el puente MQTT-LoRa y el dashboard web de monitorización.

\section*{Gateway LoRa (MKR WAN 1310)}

El gateway actúa como intermediario bidireccional entre la red LoRa y la Raspberry Pi. 
Su función principal es traducir entre el protocolo binario de LoRa y el protocolo serial estructurado que entiende el puente MQTT.

\subsection*{Funcionalidades principales}

\begin{itemize}
  \item \textbf{Recepción LoRa}: escucha continuamente mensajes de los nodos sensores y los reenvía por serial.
  \item \textbf{Transmisión LoRa}: recibe comandos por serial y los transmite a los actuadores.
  \item \textbf{Gestión de ACK}: espera confirmación del actuador y reintenta si es necesario.
  \item \textbf{Mapeo de topics}: traduce los topics internos (\texttt{sensor/0}) a topics MQTT (\texttt{sensores/puerta}).
\end{itemize}

\subsection*{Protocolo serial}

El gateway implementa un protocolo binario delimitado:

\begin{lstlisting}[language=C++, caption={Constantes del protocolo serial.}]
#define STX 0x02  // Start of Text
#define ETX 0x03  // End of Text

#define MSG_TYPE_LORA_RX  'R'  // Mensaje recibido de LoRa
#define MSG_TYPE_LORA_TX  'T'  // Mensaje a transmitir
#define MSG_TYPE_ACK      'A'  // Acknowledgment
#define MSG_TYPE_NACK     'N'  // Negative acknowledgment
#define MSG_TYPE_STATUS   'S'  // Estado del sistema
\end{lstlisting}

\subsection*{Sistema de reintentos}

Para garantizar la entrega fiable de comandos, el gateway implementa un sistema de reintentos con timeout:

\begin{lstlisting}[language=C++, caption={Configuración de reintentos.}]
const unsigned long ACK_TIMEOUT_MS = 2000;  // 2 segundos
const uint8_t MAX_ACK_RETRIES = 3;          // 3 intentos
const unsigned long TX_COOLDOWN_MS = 50;    // Cooldown entre TX
\end{lstlisting}

\section*{Puente MQTT-LoRa (Raspberry Pi)}

El script \texttt{mqtt\_lora\_bridge.py} ejecutándose en la Raspberry Pi conecta el mundo LoRa con el ecosistema MQTT.

\subsection*{Arquitectura del puente}

El puente utiliza dos hilos principales:

\begin{itemize}
  \item \textbf{Hilo serial}: lee continuamente del puerto GPIO UART y parsea las tramas del gateway.
  \item \textbf{Hilo MQTT}: gestiona la conexión al broker y procesa los mensajes suscritos.
\end{itemize}

\begin{lstlisting}[language=Python, caption={Configuración del puente.}]
SERIAL_PORT = "/dev/serial0"  # GPIO UART
SERIAL_BAUD = 115200

MQTT_BROKER = "localhost"
MQTT_PORT = 1883

# Topics de suscripcion para reenviar por LoRa
TOPICS_TO_LORA = [
    "lora/tx",
    "actuador/comando",
    "sensores/luz",
    "sensores/puerta"
]
\end{lstlisting}

\subsection*{Control de rate limiting}

Para evitar saturar el canal LoRa, el puente implementa políticas de limitación:

\begin{lstlisting}[language=Python, caption={Políticas de rate limiting.}]
MIN_TX_INTERVAL = 0.15   # Minimo 150ms entre envios
DUPLICATE_WINDOW = 2.0   # Ignorar duplicados en 2 segundos
\end{lstlisting}

\subsection*{Clases principales}

\begin{itemize}
  \item \texttt{SerialProtocol}: gestiona la comunicación serial con el gateway, incluyendo conexión, desconexión y parseo de tramas.
  \item \texttt{LoRaMessage}: estructura de datos para mensajes LoRa con campos para emisor, RSSI, SNR, topic y payload.
  \item \texttt{MQTTLoRaBridge}: clase principal que coordina serial y MQTT, implementa el mapeo de topics y las políticas de QoS.
\end{itemize}

\section*{Dashboard Web}

El dashboard proporciona una interfaz visual para monitorizar el estado de los sensores y enviar comandos manuales.

\subsection*{Tecnologías utilizadas}

\begin{itemize}
  \item \textbf{Flask}: framework web ligero para Python.
  \item \textbf{Flask-SocketIO}: extensión para comunicación en tiempo real mediante WebSockets.
  \item \textbf{Paho MQTT}: cliente MQTT para Python.
  \item \textbf{HTML/CSS/JavaScript}: interfaz de usuario responsive.
\end{itemize}

\subsection*{Arquitectura del dashboard}

\begin{figure}[H]
  \centering
  \begin{tikzpicture}[
    box/.style={draw, rounded corners, minimum width=30mm, minimum height=10mm, align=center, fill=white},
    arrow/.style={-{Stealth}, thick},
    darrow/.style={{Stealth}-{Stealth}, thick}
  ]
    \node[box, fill=blue!10] (browser) {Navegador Web};
    \node[box, fill=green!10, below=15mm of browser] (flask) {Flask + SocketIO};
    \node[box, fill=orange!10, below=15mm of flask] (mqtt) {Broker MQTT};
    
    \draw[darrow] (browser) -- node[right, font=\footnotesize]{WebSocket} (flask);
    \draw[darrow] (flask) -- node[right, font=\footnotesize]{MQTT} (mqtt);
  \end{tikzpicture}
  \caption{Arquitectura del dashboard web.}
  \label{fig:arquitectura-dashboard}
\end{figure}

\subsection*{Funcionalidades}

El dashboard ofrece las siguientes características:

\begin{itemize}
  \item \textbf{Visualización en tiempo real}: muestra el estado actual de luz y puerta con emojis e indicadores de color.
  \item \textbf{Control manual}: botones para forzar estados 0 o 1 en cada sensor/actuador.
  \item \textbf{Indicadores de conexión}: muestra el estado de la conexión WebSocket y MQTT.
  \item \textbf{Log de eventos}: registro histórico de mensajes recibidos y enviados.
\end{itemize}

\subsection*{Interfaz de usuario}

La interfaz está diseñada para ser intuitiva y accesible desde cualquier dispositivo:

\begin{itemize}
  \item Diseño responsive con CSS Grid.
  \item Tema oscuro con gradientes para reducir fatiga visual.
  \item Tarjetas con bordes redondeados y efecto glassmorphism.
  \item Emojis grandes (100px) para identificación rápida del estado.
  \item Transiciones suaves para cambios de estado.
\end{itemize}

\begin{lstlisting}[caption={Estructura de una tarjeta de sensor.}]
<div class="card">
  <div class="emoji" id="emoji-luz">💡</div>
  <div class="titulo-sensor">Sensor de Luz</div>
  <div class="estado" id="estado-luz">Desconocido</div>
  <div class="botones">
    <button onclick="publicarLuz(0)">Iluminado (0)</button>
    <button onclick="publicarLuz(1)">Oscuro (1)</button>
  </div>
</div>
\end{lstlisting}

\subsection*{Comunicación en tiempo real}

La actualización del dashboard se realiza mediante WebSockets con SocketIO:

\begin{lstlisting}[language=C, caption={Manejo de eventos SocketIO en el cliente.}]
socket.on('update_luz', function(data) {
  const valor = data.valor;
  document.getElementById('emoji-luz').textContent = 
    valor === 1 ? '🌙' : '☀️';
  document.getElementById('estado-luz').textContent = 
    valor === 1 ? 'Oscuro' : 'Iluminado';
});

socket.on('update_puerta', function(data) {
  const valor = data.valor;
  document.getElementById('emoji-puerta').textContent = 
    valor === 1 ? '🚪' : '🔒';
  document.getElementById('estado-puerta').textContent = 
    valor === 1 ? 'Abierta/Gente' : 'Cerrada/Nadie';
});
\end{lstlisting}

\subsection*{Eventos del servidor}

El servidor Flask gestiona los siguientes eventos:

\begin{table}[H]
\centering
\caption{Eventos SocketIO del dashboard.}
\label{tab:eventos-socketio}
\begin{tabular}{llp{5cm}}
\toprule
\textbf{Evento} & \textbf{Dirección} & \textbf{Descripción} \\
\midrule
\texttt{connect} & Cliente $\rightarrow$ Servidor & Cliente web conectado \\
\texttt{estado\_inicial} & Servidor $\rightarrow$ Cliente & Envía estado actual \\
\texttt{update\_luz} & Servidor $\rightarrow$ Cliente & Actualiza estado de luz \\
\texttt{update\_puerta} & Servidor $\rightarrow$ Cliente & Actualiza estado de puerta \\
\texttt{publicar\_luz} & Cliente $\rightarrow$ Servidor & Envía comando de luz \\
\texttt{publicar\_puerta} & Cliente $\rightarrow$ Servidor & Envía comando de puerta \\
\texttt{mqtt\_status} & Servidor $\rightarrow$ Cliente & Estado conexión MQTT \\
\bottomrule
\end{tabular}
\end{table}

\section*{Instalación y Ejecución}

\subsection*{Requisitos}

Los requisitos de software para ejecutar el sistema completo son:

\begin{lstlisting}[caption={Contenido de requirements.txt}]
paho-mqtt>=1.6.0
pyserial>=3.5
flask>=2.0.0
flask-socketio>=5.0.0
\end{lstlisting}

\subsection*{Configuración de la Raspberry Pi}

\begin{enumerate}
  \item Deshabilitar la consola serial en \texttt{raspi-config} y habilitar el puerto ``serial0`` para uso general.
  \item Instalar el broker Mosquitto y Python 3:
  \begin{lstlisting}[language=bash]
sudo apt install mosquitto mosquitto-clients python3-pip
  \end{lstlisting}
  \item Habilitar Mosquitto para que arranque al inicio:
  \begin{lstlisting}[language=bash]
sudo systemctl enable mosquitto
  \end{lstlisting}
  \item Instalar dependencias Python:
  \begin{lstlisting}[language=bash]
pip3 install -r requirements.txt
  \end{lstlisting}
  \item Ejecutar el puente y el dashboard:
  \begin{lstlisting}[language=bash]
python3 mqtt_lora_bridge.py &
python3 web_dashboard.py
  \end{lstlisting}
\end{enumerate}

\subsection*{Acceso al dashboard}

El dashboard es accesible desde cualquier dispositivo en la misma red local mediante:

\begin{center}
\texttt{http://<IP\_RASPBERRY>:5000}
\end{center}

La aplicación está configurada con \texttt{cors\_allowed\_origins="*"} para permitir conexiones desde cualquier 
origen durante el desarrollo.


\chapter{Conclusiones}\label{chap:conclusiones}
% Conclusiones

En este capítulo se presentan las conclusiones del proyecto, los objetivos alcanzados, las dificultades encontradas 
y las posibles líneas de mejora.

\section*{Objetivos Alcanzados}

El proyecto ha cumplido satisfactoriamente los objetivos planteados al inicio:

\begin{itemize}
  \item Se ha diseñado e implementado una arquitectura IoT completa que integra sensores físicos, comunicación LoRa, 
  un broker MQTT y aplicaciones cliente.
  
  \item Se ha mejorado el protocolo maestro-esclavo de la practica 2 para la lectura de sensores ultrasónicos (SRF01/SRF02) 
  y analógicos (LDR).
  
  \item Se ha implementado un formato de tramas LoRa con direccionamiento, identificadores de mensaje y sistema de ACK 
  que garantiza la entrega fiable de comandos.
  
  \item Se ha desplegado un broker MQTT (Mosquitto) y un puente serie-MQTT que expone los datos de los sensores y permite 
  el control del actuador desde aplicaciones externas.
  
  \item Se ha desarrollado un dashboard web interactivo que visualiza en tiempo real el estado de luz y puerta, y 
  permite enviar comandos manuales.
  
  \item El sistema responde automáticamente a los cambios detectados por los sensores, encendiendo o apagando la luz y 
  abriendo o cerrando la puerta según los umbrales configurados.
\end{itemize}

\section*{Dificultades Encontradas}

Durante el desarrollo del proyecto se encontraron las siguientes dificultades:

\begin{itemize}
  \item \textbf{Coordinación del equipo}: al tratarse de un proyecto multidisciplinar, fue necesario coordinar las tareas entre los miembros del equipo para asegurar una 
  integración fluida de los distintos componentes. La principal dificultad fue alinear los tiempos de desarrollo y pruebas entre los 
  distintos miembros.
  
  \item \textbf{Sincronización de parámetros LoRa}: todos los nodos deben utilizar exactamente la misma configuración de frecuencia,
   ancho de banda, spreading factor y sync word. Pequeñas discrepancias impiden la comunicación.
  
  \item \textbf{Tiempos de propagación LoRa}: con SF10 y BW 62.5~kHz, el tiempo de transmisión de un paquete puede superar varios 
  cientos de milisegundos, lo que requiere ajustar los timeouts de ACK y las políticas de rate limiting.
  
\end{itemize}

\section*{Conocimientos Aplicados}

El proyecto ha permitido aplicar de forma práctica conceptos de diversas áreas:

\begin{itemize}
  \item \textbf{Comunicaciones inalámbricas}: configuración de módulos LoRa, comprensión de parámetros como spreading factor, 
  ancho de banda y coding rate.
  
  \item \textbf{Protocolos de comunicación}: diseño de protocolos binarios compactos, sistemas de ACK/NACK, y manejo de duplicados.
  
  \item \textbf{Arquitectura IoT}: separación de responsabilidades entre capas (sensores, transporte, integración, presentación).
  
  \item \textbf{Sistemas embebidos}: programación de Arduino con restricciones de memoria y procesamiento, uso de interrupciones 
  y temporizadores.
  
  \item \textbf{Desarrollo web}: creación de aplicaciones en tiempo real con Flask, SocketIO y WebSockets.
  
  \item \textbf{Integración de sistemas}: conexión de componentes heterogéneos mediante protocolos estándar (MQTT, Serial).
\end{itemize}

\section*{Líneas de Mejora}

El sistema actual es funcional y cumple los requisitos académicos, pero existen varias áreas de mejora para un despliegue en producción:

\begin{itemize}
  \item \textbf{Seguridad}: implementar cifrado en la capa de aplicación LoRa y autenticación MQTT con TLS.
  
  \item \textbf{Persistencia}: almacenar el histórico de lecturas en una base de datos (InfluxDB, SQLite) para análisis posterior.
  
  \item \textbf{Escalabilidad}: soportar múltiples actuadores y sensores con descubrimiento automático de nodos.
  
  \item \textbf{Interfaz móvil}: desarrollar una aplicación nativa o PWA para control desde dispositivos móviles.
  
  \item \textbf{Integración con asistentes}: conectar con Home Assistant, Google Home o Alexa para control por voz.
  
  \item \textbf{Bajo consumo}: implementar modos de sleep en los nodos LoRa para maximizar la duración de baterías.
  
  \item \textbf{Redundancia}: añadir un segundo gateway para tolerancia a fallos.
  
  \item \textbf{Monitorización}: integrar métricas de sistema (RSSI, SNR, latencia) en un dashboard de operaciones.
\end{itemize}

\section*{Valoración Personal}

El desarrollo de este proyecto ha sido una experiencia enriquecedora que ha permitido integrar conocimientos de múltiples
 asignaturas del grado. La combinación de hardware embebido, comunicaciones inalámbricas y desarrollo web ofrece una visión completa 
 del ecosistema IoT.

El trabajo en equipo ha sido fundamental para abordar las distintas áreas del proyecto, distribuyendo las tareas según las fortalezas 
de cada miembro y manteniendo una comunicación fluida para integrar los componentes.

El sistema resultante, aunque sencillo en su funcionalidad (control de luz y puerta), demuestra que es posible construir soluciones 
domóticas completas con hardware de bajo coste y software libre, sentando las bases para proyectos más ambiciosos en el futuro.


% ---- Bibliografía (opcional) ----
\begin{thebibliography}{9}

\bibitem{raspberrypi}
Raspberry Pi Foundation.
\newblock \emph{Raspberry Pi Documentation}.
\newblock Disponible en: \url{https://www.raspberrypi.com/documentation/computers/raspberry-pi.html}.
\newblock Consultado a 26 enero de 2026.

\bibitem{ldr-arduino}
ElectronicsFan123.
\newblock \emph{Interfacing Arduino UNO with LDR}.
\newblock Project Hub, Arduino.
\newblock Disponible en: \url{https://projecthub.arduino.cc/electronicsfan123/interfacing-arduino-uno-with-ldr-61f455}.
\newblock Consultado a 26 enero de 2026.

\bibitem{kicad}
KiCad Project.
\newblock \emph{KiCad EDA}.
\newblock Disponible en: \url{https://www.kicad.org/}.
\newblock Consultado a 26 enero de 2026.

\bibitem{kicad-docs}
KiCad Project.
\newblock \emph{KiCad Documentation}.
\newblock Disponible en: \url{https://docs.kicad.org/}.
\newblock Consultado a 26 enero de 2026.

\bibitem{SRF02-docs}
Robot Electronics.
\newblock \emph{SRF02 Documentation}.
\newblock Disponible en: \url{https://www.robot-electronics.co.uk/htm/srf02tech.htm}.
\newblock Consultado a 26 enero de 2026.


\end{thebibliography}

\end{document}
