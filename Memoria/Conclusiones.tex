% Conclusiones

En este capítulo se presentan las conclusiones del proyecto, los objetivos alcanzados, las dificultades encontradas 
y las posibles líneas de mejora.

\section*{Objetivos Alcanzados}

El proyecto ha cumplido satisfactoriamente los objetivos planteados al inicio:

\begin{itemize}
  \item Se ha diseñado e implementado una arquitectura IoT completa que integra sensores físicos, comunicación LoRa, 
  un broker MQTT y aplicaciones cliente.
  
  \item Se ha mejorado el protocolo maestro-esclavo de la practica 2 para la lectura de sensores ultrasónicos (SRF01/SRF02) 
  y analógicos (LDR).
  
  \item Se ha implementado un formato de tramas LoRa con direccionamiento, identificadores de mensaje y sistema de ACK 
  que garantiza la entrega fiable de comandos.
  
  \item Se ha desplegado un broker MQTT (Mosquitto) y un puente serie-MQTT que expone los datos de los sensores y permite 
  el control del actuador desde aplicaciones externas.
  
  \item Se ha desarrollado un dashboard web interactivo que visualiza en tiempo real el estado de luz y puerta, y 
  permite enviar comandos manuales.
  
  \item El sistema responde automáticamente a los cambios detectados por los sensores, encendiendo o apagando la luz y 
  abriendo o cerrando la puerta según los umbrales configurados.
\end{itemize}

\section*{Dificultades Encontradas}

Durante el desarrollo del proyecto se encontraron las siguientes dificultades:

\begin{itemize}
  \item \textbf{Coordinación del equipo}: al tratarse de un proyecto multidisciplinar, fue necesario coordinar las tareas entre los miembros del equipo para asegurar una 
  integración fluida de los distintos componentes. La principal dificultad fue alinear los tiempos de desarrollo y pruebas entre los 
  distintos miembros.
  
  \item \textbf{Sincronización de parámetros LoRa}: todos los nodos deben utilizar exactamente la misma configuración de frecuencia,
   ancho de banda, spreading factor y sync word. Pequeñas discrepancias impiden la comunicación.
  
  \item \textbf{Tiempos de propagación LoRa}: con SF10 y BW 62.5~kHz, el tiempo de transmisión de un paquete puede superar varios 
  cientos de milisegundos, lo que requiere ajustar los timeouts de ACK y las políticas de rate limiting.
  
\end{itemize}

\section*{Conocimientos Aplicados}

El proyecto ha permitido aplicar de forma práctica conceptos de diversas áreas:

\begin{itemize}
  \item \textbf{Comunicaciones inalámbricas}: configuración de módulos LoRa, comprensión de parámetros como spreading factor, 
  ancho de banda y coding rate.
  
  \item \textbf{Protocolos de comunicación}: diseño de protocolos binarios compactos, sistemas de ACK/NACK, y manejo de duplicados.
  
  \item \textbf{Arquitectura IoT}: separación de responsabilidades entre capas (sensores, transporte, integración, presentación).
  
  \item \textbf{Sistemas embebidos}: programación de Arduino con restricciones de memoria y procesamiento, uso de interrupciones 
  y temporizadores.
  
  \item \textbf{Desarrollo web}: creación de aplicaciones en tiempo real con Flask, SocketIO y WebSockets.
  
  \item \textbf{Integración de sistemas}: conexión de componentes heterogéneos mediante protocolos estándar (MQTT, Serial).
\end{itemize}

\section*{Líneas de Mejora}

El sistema actual es funcional y cumple los requisitos académicos, pero existen varias áreas de mejora para un despliegue en producción:

\begin{itemize}
  \item \textbf{Seguridad}: implementar cifrado en la capa de aplicación LoRa y autenticación MQTT con TLS.
  
  \item \textbf{Persistencia}: almacenar el histórico de lecturas en una base de datos (InfluxDB, SQLite) para análisis posterior.
  
  \item \textbf{Escalabilidad}: soportar múltiples actuadores y sensores con descubrimiento automático de nodos.
  
  \item \textbf{Interfaz móvil}: desarrollar una aplicación nativa o PWA para control desde dispositivos móviles.
  
  \item \textbf{Integración con asistentes}: conectar con Home Assistant, Google Home o Alexa para control por voz.
  
  \item \textbf{Bajo consumo}: implementar modos de sleep en los nodos LoRa para maximizar la duración de baterías.
  
  \item \textbf{Redundancia}: añadir un segundo gateway para tolerancia a fallos.
  
  \item \textbf{Monitorización}: integrar métricas de sistema (RSSI, SNR, latencia) en un dashboard de operaciones.
\end{itemize}

\section*{Valoración Personal}

El desarrollo de este proyecto ha sido una experiencia enriquecedora que ha permitido integrar conocimientos de múltiples
 asignaturas del grado. La combinación de hardware embebido, comunicaciones inalámbricas y desarrollo web ofrece una visión completa 
 del ecosistema IoT.

El trabajo en equipo ha sido fundamental para abordar las distintas áreas del proyecto, distribuyendo las tareas según las fortalezas 
de cada miembro y manteniendo una comunicación fluida para integrar los componentes.

El sistema resultante, aunque sencillo en su funcionalidad (control de luz y puerta), demuestra que es posible construir soluciones 
domóticas completas con hardware de bajo coste y software libre, sentando las bases para proyectos más ambiciosos en el futuro.
