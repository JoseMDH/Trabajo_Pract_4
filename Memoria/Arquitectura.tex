% Arquitectura

En este capítulo se describe la arquitectura global del sistema, los flujos de comunicación entre los distintos nodos y los protocolos 
utilizados en cada capa.

\section*{Visión General}

El sistema sigue una arquitectura distribuida de tipo \emph{sensor-gateway-actuador} con integración en la nube a través de MQTT. 
Los componentes principales son:

\begin{enumerate}
  \item \textbf{Nodo de sensores}: compuesto por un esclavo (Arduino MKR WAN 1310) y un maestro (MKR WAN 1310).
  \item \textbf{Gateway LoRa}: Arduino MKR WAN 1310 conectado a la Raspberry Pi.
  \item \textbf{Raspberry Pi}: ejecuta el broker MQTT y el puente serie-MQTT.
  \item \textbf{2 Fuentes de alimentación}: proporcionan energía a los sensores y dispositivos.
  \item \textbf{Nodo actuador}: Arduino MKR WAN 1310 conectado a los motores.
  \item \textbf{Aplicaciones cliente}: dashboard web.
\end{enumerate}

\section*{Diagrama de Flujo de Datos}

El flujo de datos desde los sensores hasta las aplicaciones cliente sigue el siguiente recorrido:

\begin{figure}[H]
  \centering
  \begin{tikzpicture}[
    box/.style={draw, rounded corners, minimum width=28mm, minimum height=10mm,
                align=center, fill=blue!10},
    arrow/.style={-{Stealth}, thick},
    node distance=18mm
  ]
    % Fila superior
    \node[box] (sensor)  {Sensores\\(LDR, SRF)};
    \node[box, right=30mm of sensor] (esclavo) {Esclavo};
    \node[box, right=30mm of esclavo] (maestro) {Maestro};

    % Fila inferior (más ancha, hacia la izquierda)
    \node[box, below=25mm of esclavo, xshift=15mm] (gateway) {Gateway};
    \node[box, left=35mm of gateway] (raspi) {Raspberry Pi};

    % Flechas fila superior
    \draw[arrow] (sensor) -- node[above, font=\scriptsize]{Analógico/I2C} (esclavo);
    \draw[arrow] (esclavo) -- node[above, font=\scriptsize]{Serial} (maestro);

    % Flecha Maestro -> Gateway (diagonal hacia abajo-izquierda)
    \draw[arrow, dashed] (maestro.south) -- node[sloped, above, font=\scriptsize]{LoRa}
        (gateway.north east);

    % Flecha Gateway -> Raspberry (horizontal hacia la izquierda)
    \draw[arrow] (gateway.west) -- node[above, font=\scriptsize]{Serial (GPIO UART)}
        (raspi.east);
  \end{tikzpicture}
  \caption{Flujo de datos desde sensores, pasando por el maestro, hasta la Raspberry Pi.}
  \label{fig:flujo-sensores}
\end{figure}


\section*{Protocolo Maestro-Esclavo (Serial)}

La comunicación entre el esclavo y el maestro utiliza un protocolo serie sencillo basado en tramas estructuradas. 
El esclavo envía periódicamente las lecturas de los sensores al maestro.

\begin{table}[H]
\centering
\caption{Estructura de respuesta del esclavo al maestro.}
\label{tab:protocolo-esclavo}
\begin{tabular}{lll}
\toprule
\textbf{Campo} & \textbf{Tamaño} & \textbf{Descripción} \\
\midrule
Código respuesta & 1 byte & Tipo de dato (distancia, luz, error) \\
ID Sensor & 1 byte & Identificador del sensor \\
Dato (MSB) & 1 byte & Byte alto de la medida \\
Dato (LSB) & 1 byte & Byte bajo de la medida \\
\bottomrule
\end{tabular}
\end{table}

\section*{Protocolo LoRa}

Las comunicaciones LoRa entre el maestro, el gateway y el actuador utilizan un formato de paquete común con cabecera explícita:

\begin{table}[H]
\centering
\caption{Formato de paquete LoRa.}
\label{tab:formato-lora}
\begin{tabular}{llp{6cm}}
\toprule
\textbf{Campo} & \textbf{Tamaño} & \textbf{Descripción} \\
\midrule
Destino & 1 byte & Dirección del nodo destino \\
Origen & 1 byte & Dirección del nodo emisor \\
Msg ID (MSB) & 1 byte & Identificador de mensaje (byte alto) \\
Msg ID (LSB) & 1 byte & Identificador de mensaje (byte bajo) \\
Longitud & 1 byte & Tamaño del payload en bytes \\
Payload & N bytes & Datos del mensaje \\
\bottomrule
\end{tabular}
\end{table}

Las direcciones asignadas a cada nodo son:

\begin{table}[H]
\centering
\caption{Direcciones LoRa de los nodos del sistema.}
\label{tab:direcciones-lora}
\begin{tabular}{ll}
\toprule
\textbf{Nodo} & \textbf{Dirección} \\
\midrule
Maestro (sensores) & \texttt{0x04} \\
Gateway & \texttt{0x05} \\
Actuador & \texttt{0x06} \\
Broadcast & \texttt{0xFF} \\
\bottomrule
\end{tabular}
\end{table}

\section*{Protocolo Serial Gateway-Raspberry}

El gateway y la Raspberry Pi se comunican mediante un protocolo binario delimitado por caracteres STX/ETX:

\begin{table}[H]
\centering
\caption{Formato de trama serial entre Gateway y Raspberry Pi.}
\label{tab:protocolo-serial-gateway}
\begin{tabular}{llp{5cm}}
\toprule
\textbf{Campo} & \textbf{Valor/Tamaño} & \textbf{Descripción} \\
\midrule
STX & \texttt{0x02} & Inicio de trama \\
Tipo & 1 byte & R=RX, T=TX, A=ACK, N=NACK, S=Status \\
Topic Len & 1 byte & Longitud del topic \\
Topic & N bytes & Nombre del topic \\
Payload Len & 1 byte & Longitud del payload \\
Payload & N bytes & Datos \\
ETX & \texttt{0x03} & Fin de trama \\
\bottomrule
\end{tabular}
\end{table}

\section*{Integración MQTT}

El broker MQTT (Mosquitto) en la Raspberry Pi expone los siguientes topics:

\begin{table}[H]
\centering
\caption{Topics MQTT del sistema.}
\label{tab:topics-mqtt}
\begin{tabular}{llp{5cm}}
\toprule
\textbf{Topic} & \textbf{Dirección} & \textbf{Descripción} \\
\midrule
\texttt{sensores/luz} & Publicación & Estado del sensor de luz (0/1) \\
\texttt{sensores/puerta} & Publicación & Estado del sensor de proximidad (0/1) \\
\texttt{lora/rx} & Publicación & Mensajes LoRa en bruto (JSON) \\
\texttt{lora/tx} & Suscripción & Enviar mensaje LoRa genérico \\
\texttt{actuador/comando} & Suscripción & Comandos para el actuador \\
\bottomrule
\end{tabular}
\end{table}

El puente MQTT-LoRa (\texttt{mqtt\_lora\_bridge.py}) realiza el mapeo entre los topics internos de LoRa y los topics MQTT legibles:

\begin{itemize}
  \item \texttt{sensor/0} $\rightarrow$ \texttt{sensores/puerta}
  \item \texttt{sensor/1} $\rightarrow$ \texttt{sensores/luz}
\end{itemize}

\section*{Flujo de Control (Actuador)}

Cuando una aplicación cliente desea controlar el actuador, el flujo es el siguiente:

\begin{enumerate}
  \item La aplicación publica en \texttt{sensores/luz} o \texttt{sensores/puerta}.
  \item El bridge recibe el mensaje MQTT y lo convierte a trama serial.
  \item El gateway recibe la trama y la transmite por LoRa al actuador.
  \item El actuador responde con ACK y ejecuta la acción.
  \item El gateway recibe el ACK y notifica al bridge.
  \item Si no hay ACK en el tiempo límite, el gateway reintenta (hasta 3 veces).
\end{enumerate}

\section*{Políticas de Calidad de Servicio}

Para garantizar la fiabilidad del sistema se implementan las siguientes políticas:

\begin{itemize}
  \item \textbf{Rate limiting}: el bridge limita los envíos a un mínimo de 150~ms entre mensajes para no saturar LoRa.
  \item \textbf{Eliminación de duplicados}: mensajes idénticos recibidos en una ventana de 2 segundos se ignoran.
  \item \textbf{Reintentos con ACK}: el gateway reintenta hasta 3 veces si no recibe confirmación del actuador.
  \item \textbf{Timeout configurable}: el tiempo máximo de espera de ACK es de 2 segundos, ajustado para SF10 y BW 62.5~kHz.
\end{itemize}

\section*{Diagrama de Secuencia}

A continuación se muestra el diagrama de secuencia para un ciclo completo de detección y actuación:

\begin{figure}[H]
  \centering
  \begin{tikzpicture}[
    node distance=18mm,
    actor/.style={draw, minimum width=18mm, minimum height=8mm, align=center},
    arrow/.style={-{Stealth}, thick},
    darrow/.style={-{Stealth}, thick, dashed}
  ]
    % Actores
    \node[actor] (esc) {Esclavo};
    \node[actor, right=of esc] (mae) {Maestro};
    \node[actor, right=of mae] (gw) {Gateway};
    \node[actor, right=of gw] (rpi) {Raspberry};
    \node[actor, right=of rpi] (act) {Actuador};
    
    % Líneas de vida
    \foreach \n in {esc, mae, gw, rpi, act} {
      \draw[dashed, gray] (\n.south) -- ++(0,-55mm);
    }
    
    % Mensajes
    \draw[arrow] ([yshift=-10mm]esc.south) -- node[above, font=\scriptsize]{Lectura sensor} ([yshift=-10mm]mae.south);
    \draw[darrow] ([yshift=-18mm]mae.south) -- node[above, font=\scriptsize]{LoRa: sensor/1} ([yshift=-18mm]gw.south);
    \draw[arrow] ([yshift=-26mm]gw.south) -- node[above, font=\scriptsize]{Serial} ([yshift=-26mm]rpi.south);
    \draw[arrow] ([yshift=-34mm]rpi.south) -- node[above, font=\scriptsize]{Serial: cmd} ([yshift=-34mm]gw.south);
    \draw[darrow] ([yshift=-42mm]gw.south) -- node[above, font=\scriptsize]{LoRa: cmd} ([yshift=-42mm]act.south);
    \draw[darrow] ([yshift=-50mm]act.south) -- node[above, font=\scriptsize]{LoRa: ACK} ([yshift=-50mm]gw.south);
  \end{tikzpicture}
  \caption{Diagrama de secuencia: detección de luz y actuación.}
  \label{fig:secuencia-luz}
\end{figure}

\section*{Consideraciones de Seguridad}

Aunque el sistema no implementa cifrado en las comunicaciones LoRa (fuera del alcance de este proyecto académico), 
se aplican medidas básicas:

\begin{itemize}
  \item Filtrado por dirección de origen en el actuador.
  \item Sync Word compartido (\texttt{0x12}) que actúa como identificador de red.
  \item Comunicación MQTT local (localhost) sin exposición a Internet.
\end{itemize}

En un despliegue real se debería añadir cifrado en la capa de aplicación y autenticación MQTT con usuario/contraseña o certificados.
