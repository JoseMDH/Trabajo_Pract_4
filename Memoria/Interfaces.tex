% Interfaces - Broker, Dashboard y Gateway

En este capítulo se describen las interfaces de software del sistema: el gateway LoRa, 
el puente MQTT-LoRa y el dashboard web de monitorización.

\section*{Gateway LoRa (MKR WAN 1310)}

El gateway actúa como intermediario bidireccional entre la red LoRa y la Raspberry Pi. 
Su función principal es traducir entre el protocolo binario de LoRa y el protocolo serial estructurado que entiende el puente MQTT.

\subsection*{Funcionalidades principales}

\begin{itemize}
  \item \textbf{Recepción LoRa}: escucha continuamente mensajes de los nodos sensores y los reenvía por serial.
  \item \textbf{Transmisión LoRa}: recibe comandos por serial y los transmite a los actuadores.
  \item \textbf{Gestión de ACK}: espera confirmación del actuador y reintenta si es necesario.
  \item \textbf{Mapeo de topics}: traduce los topics internos (\texttt{sensor/0}) a topics MQTT (\texttt{sensores/puerta}).
\end{itemize}

\subsection*{Protocolo serial}

El gateway implementa un protocolo binario delimitado:

\begin{lstlisting}[language=C++, caption={Constantes del protocolo serial.}]
#define STX 0x02  // Start of Text
#define ETX 0x03  // End of Text

#define MSG_TYPE_LORA_RX  'R'  // Mensaje recibido de LoRa
#define MSG_TYPE_LORA_TX  'T'  // Mensaje a transmitir
#define MSG_TYPE_ACK      'A'  // Acknowledgment
#define MSG_TYPE_NACK     'N'  // Negative acknowledgment
#define MSG_TYPE_STATUS   'S'  // Estado del sistema
\end{lstlisting}

\subsection*{Sistema de reintentos}

Para garantizar la entrega fiable de comandos, el gateway implementa un sistema de reintentos con timeout:

\begin{lstlisting}[language=C++, caption={Configuración de reintentos.}]
const unsigned long ACK_TIMEOUT_MS = 2000;  // 2 segundos
const uint8_t MAX_ACK_RETRIES = 3;          // 3 intentos
const unsigned long TX_COOLDOWN_MS = 50;    // Cooldown entre TX
\end{lstlisting}

\section*{Puente MQTT-LoRa (Raspberry Pi)}

El script \texttt{mqtt\_lora\_bridge.py} ejecutándose en la Raspberry Pi conecta el mundo LoRa con el ecosistema MQTT.

\subsection*{Arquitectura del puente}

El puente utiliza dos hilos principales:

\begin{itemize}
  \item \textbf{Hilo serial}: lee continuamente del puerto GPIO UART y parsea las tramas del gateway.
  \item \textbf{Hilo MQTT}: gestiona la conexión al broker y procesa los mensajes suscritos.
\end{itemize}

\begin{lstlisting}[language=Python, caption={Configuración del puente.}]
SERIAL_PORT = "/dev/serial0"  # GPIO UART
SERIAL_BAUD = 115200

MQTT_BROKER = "localhost"
MQTT_PORT = 1883

# Topics de suscripcion para reenviar por LoRa
TOPICS_TO_LORA = [
    "lora/tx",
    "actuador/comando",
    "sensores/luz",
    "sensores/puerta"
]
\end{lstlisting}

\subsection*{Control de rate limiting}

Para evitar saturar el canal LoRa, el puente implementa políticas de limitación:

\begin{lstlisting}[language=Python, caption={Políticas de rate limiting.}]
MIN_TX_INTERVAL = 0.15   # Minimo 150ms entre envios
DUPLICATE_WINDOW = 2.0   # Ignorar duplicados en 2 segundos
\end{lstlisting}

\subsection*{Clases principales}

\begin{itemize}
  \item \texttt{SerialProtocol}: gestiona la comunicación serial con el gateway, incluyendo conexión, desconexión y parseo de tramas.
  \item \texttt{LoRaMessage}: estructura de datos para mensajes LoRa con campos para emisor, RSSI, SNR, topic y payload.
  \item \texttt{MQTTLoRaBridge}: clase principal que coordina serial y MQTT, implementa el mapeo de topics y las políticas de QoS.
\end{itemize}

\section*{Dashboard Web}

El dashboard proporciona una interfaz visual para monitorizar el estado de los sensores y enviar comandos manuales.

\subsection*{Tecnologías utilizadas}

\begin{itemize}
  \item \textbf{Flask}: framework web ligero para Python.
  \item \textbf{Flask-SocketIO}: extensión para comunicación en tiempo real mediante WebSockets.
  \item \textbf{Paho MQTT}: cliente MQTT para Python.
  \item \textbf{HTML/CSS/JavaScript}: interfaz de usuario responsive.
\end{itemize}

\subsection*{Arquitectura del dashboard}

\begin{figure}[H]
  \centering
  \begin{tikzpicture}[
    box/.style={draw, rounded corners, minimum width=30mm, minimum height=10mm, align=center, fill=white},
    arrow/.style={-{Stealth}, thick},
    darrow/.style={{Stealth}-{Stealth}, thick}
  ]
    \node[box, fill=blue!10] (browser) {Navegador Web};
    \node[box, fill=green!10, below=15mm of browser] (flask) {Flask + SocketIO};
    \node[box, fill=orange!10, below=15mm of flask] (mqtt) {Broker MQTT};
    
    \draw[darrow] (browser) -- node[right, font=\footnotesize]{WebSocket} (flask);
    \draw[darrow] (flask) -- node[right, font=\footnotesize]{MQTT} (mqtt);
  \end{tikzpicture}
  \caption{Arquitectura del dashboard web.}
  \label{fig:arquitectura-dashboard}
\end{figure}

\subsection*{Funcionalidades}

El dashboard ofrece las siguientes características:

\begin{itemize}
  \item \textbf{Visualización en tiempo real}: muestra el estado actual de luz y puerta con emojis e indicadores de color.
  \item \textbf{Control manual}: botones para forzar estados 0 o 1 en cada sensor/actuador.
  \item \textbf{Indicadores de conexión}: muestra el estado de la conexión WebSocket y MQTT.
  \item \textbf{Log de eventos}: registro histórico de mensajes recibidos y enviados.
\end{itemize}

\subsection*{Interfaz de usuario}

La interfaz está diseñada para ser intuitiva y accesible desde cualquier dispositivo:

\begin{itemize}
  \item Diseño responsive con CSS Grid.
  \item Tema oscuro con gradientes para reducir fatiga visual.
  \item Tarjetas con bordes redondeados y efecto glassmorphism.
  \item Emojis grandes (100px) para identificación rápida del estado.
  \item Transiciones suaves para cambios de estado.
\end{itemize}

\begin{lstlisting}[caption={Estructura de una tarjeta de sensor.}]
<div class="card">
  <div class="emoji" id="emoji-luz">💡</div>
  <div class="titulo-sensor">Sensor de Luz</div>
  <div class="estado" id="estado-luz">Desconocido</div>
  <div class="botones">
    <button onclick="publicarLuz(0)">Iluminado (0)</button>
    <button onclick="publicarLuz(1)">Oscuro (1)</button>
  </div>
</div>
\end{lstlisting}

\subsection*{Comunicación en tiempo real}

La actualización del dashboard se realiza mediante WebSockets con SocketIO:

\begin{lstlisting}[language=C, caption={Manejo de eventos SocketIO en el cliente.}]
socket.on('update_luz', function(data) {
  const valor = data.valor;
  document.getElementById('emoji-luz').textContent = 
    valor === 1 ? '🌙' : '☀️';
  document.getElementById('estado-luz').textContent = 
    valor === 1 ? 'Oscuro' : 'Iluminado';
});

socket.on('update_puerta', function(data) {
  const valor = data.valor;
  document.getElementById('emoji-puerta').textContent = 
    valor === 1 ? '🚪' : '🔒';
  document.getElementById('estado-puerta').textContent = 
    valor === 1 ? 'Abierta/Gente' : 'Cerrada/Nadie';
});
\end{lstlisting}

\subsection*{Eventos del servidor}

El servidor Flask gestiona los siguientes eventos:

\begin{table}[H]
\centering
\caption{Eventos SocketIO del dashboard.}
\label{tab:eventos-socketio}
\begin{tabular}{llp{5cm}}
\toprule
\textbf{Evento} & \textbf{Dirección} & \textbf{Descripción} \\
\midrule
\texttt{connect} & Cliente $\rightarrow$ Servidor & Cliente web conectado \\
\texttt{estado\_inicial} & Servidor $\rightarrow$ Cliente & Envía estado actual \\
\texttt{update\_luz} & Servidor $\rightarrow$ Cliente & Actualiza estado de luz \\
\texttt{update\_puerta} & Servidor $\rightarrow$ Cliente & Actualiza estado de puerta \\
\texttt{publicar\_luz} & Cliente $\rightarrow$ Servidor & Envía comando de luz \\
\texttt{publicar\_puerta} & Cliente $\rightarrow$ Servidor & Envía comando de puerta \\
\texttt{mqtt\_status} & Servidor $\rightarrow$ Cliente & Estado conexión MQTT \\
\bottomrule
\end{tabular}
\end{table}

\section*{Instalación y Ejecución}

\subsection*{Requisitos}

Los requisitos de software para ejecutar el sistema completo son:

\begin{lstlisting}[caption={Contenido de requirements.txt}]
paho-mqtt>=1.6.0
pyserial>=3.5
flask>=2.0.0
flask-socketio>=5.0.0
\end{lstlisting}

\subsection*{Configuración de la Raspberry Pi}

\begin{enumerate}
  \item Deshabilitar la consola serial en \texttt{raspi-config} y habilitar el puerto ``serial0`` para uso general.
  \item Instalar el broker Mosquitto y Python 3:
  \begin{lstlisting}[language=bash]
sudo apt install mosquitto mosquitto-clients python3-pip
  \end{lstlisting}
  \item Habilitar Mosquitto para que arranque al inicio:
  \begin{lstlisting}[language=bash]
sudo systemctl enable mosquitto
  \end{lstlisting}
  \item Instalar dependencias Python:
  \begin{lstlisting}[language=bash]
pip3 install -r requirements.txt
  \end{lstlisting}
  \item Ejecutar el puente y el dashboard:
  \begin{lstlisting}[language=bash]
python3 mqtt_lora_bridge.py &
python3 web_dashboard.py
  \end{lstlisting}
\end{enumerate}

\subsection*{Acceso al dashboard}

El dashboard es accesible desde cualquier dispositivo en la misma red local mediante:

\begin{center}
\texttt{http://<IP\_RASPBERRY>:5000}
\end{center}

La aplicación está configurada con \texttt{cors\_allowed\_origins="*"} para permitir conexiones desde cualquier 
origen durante el desarrollo.
